\chapter{Conclusion}
In conclusion, this project focused on the development of a social media platform called ConnectSphere. The project's development included enhancing user engagement through the implementation of different features and functions, while ensuring a secure user authentication and dynamic content support. The overall goal was to create a dynamic social media application where users could communicate, share content and connect with others. During the development process, several goals were successfully achieved. Features like posts, likes, dislikes, videos and live chat have improved the user engagement. Token-based methods for login and registrations was used to implement secure user authentication. The application supports dynamic content updates and deletions for user generated posts. The user can follow and unfollow other users on the platform, as well as a recommendation system that helps users in find new contacts. A search function is implemented to allow users to find other users by username, with responsive search results. User profiles are provided to view important details and follow counts. Learning and applying the best practices with technologies like MERN, JWT, Socket.IO and Redux, gaining valuable project building skills. 
\newline \newline
The System Evaluation chapter highlights the strengths and weakness of the system, including the success of secure user authentication, database management with MongoDB and frontend and backend development. Challenges encountered during deployment, such as resolving URL issues for HTTP request, were effectively addressed, leading to a successful deployment to Render. Implementing secure or token-based authentication ensured secure user login and registration, protecting user data and accounts from unauthorized access. JSON web token is the technology is that helps the application apply security to the social media platform. With the token there is a way to know who has access to the application, what their role is and which links have access \cite{haekal2016token}. This add an extra layer of protection and improves the social media platform's strengths.
\newline \newline
During deployment, the system faces challenges including resolving URL issues for HTTP request when transitioning from local development environments to production environment. This deployment issue took multiple attempts to resolve out the problem. The challenges encountered were resolved once the problems were identified. ConnectSphere is successfully deployed to allow users to share content and interact with other users. Scalability is a strong point of the web application. The system has demonstrated scalability in handling a growing user base and content, and offers potential for further expansion and future enhancement. MongoDB is a reliable database that has allowed ConnectSphere to achieve this in a system positive way. MongoDB provides excellent performance and scalability to handle the enormous complexity of web development and usage \cite{aryal2020mern}. Mongoose worked well together with MongoDB because it provided functionality to the features in the database including query building capabilities and business logic in the data \cite{aryal2020mern}. The integration of Socket.IO has enabled the web application to increase user engagement. The features of notifications and live chat features have enabled the social media platform to present dynamic content. The integration with Socket.IO enabled real-time communication and updates to improve the functionality and user experience of the platform.
\newline \newline
While the system generally performed well, there may be areas for performance optimization, particularly when handling a large number of concurrent user interactions or content updates. The platform offers various features for user interaction such as likes, posts, dislikes, and comments, but additional refinements of the user interface and experience could improve usability and engagement. The main aim of the web application is to provide users with an impressive platform to enjoy. While test coverage was performed using tools such as Postman, there may have been areas where test coverage could be expanded to ensure complete validation of the system functionality and performance. There are many error handling inserted into the code to avoid possible problems such as insufficient feedback to users in case of errors. Testing in the frontend development was challenging components had to be carefully integrated to achieve the goal of the web application. 
\newline \newline
React and Material-UI are the two main technologies that played a crucial role in the delivering the frontend application. The user interface has been developed efficiently using React. The component based architecture in React contributed to a responsive and interactive user experience. In order to design the capabilities of the posts to obtain dynamic support, various components have been developed. Material-UI was used to develop the user interface components, providing a rich set of customizable user interface elements that bring consistency and aesthetic appeal to the web application. The fresh design make the social media platform look creative and intuitive. Material-UI has significantly improved the user experience in ConnectSphere. Other frontend technologies such as Redux, Formik and Yup, React Router and Dropzone also played an important role in the creation of ConnectSphere. Redux efficiently handles state management and works with local storage. Formik and Yup handled the form validation of login and registration. Dropzone allows you to upload image and video. React Router was responsible for navigation in the web application. Node.js and Express.Js are the two main technologies responsible for the backend of the project. The development of API, middleware and request paths has been simplified through the use of Node.js and Express. MongoDB is where the data for the code is stored, accommodating the dynamic content requirements of the web application. Overall, the project demonstrated the ability to create a functional and engaging social media platform while overcoming various technical challenges. There is potential for additional refinements and improvements in the future to expand the capabilities of the platform and improve the user experience.