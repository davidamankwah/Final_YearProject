\chapter{Methodology}
The goal of the project was to build a social media platform that would allow users to enjoy content and interact in a meaningful ways. Users can share posts from other users. The MERN stack of MongoDB, Express.js, React.js and Node.js, is used to develop the social media platform. The growing popularity of social media platforms focused on user engagement is the main reason to start the social media project. ConnectSphere was created using an Agile software development methodology. The Agile approach allows for an iterative development cycle. The objective was to create a community where users to could communicate through various features such as posts, comments and messages. 

The project involved trying to create a social media application from scratch and developing project creation skills. Learn about useful approaches for a MERN stack application using various technologies. While coding the project, react router, formik and yup, JSON web token, dropzone, redux toolkit, and multer were all learned for the front-end and back-end implementation. The back-end was initially implemented to configure middle-ware, controllers, models, routes and database connections. Once the back-end was fully implemented, the front-end configures widgets, components and pages. The front-end encountered some difficulties during the design of the project. The back-end was less difficult but there were few a difficult encounters. This sections provides an overview of the methodologies employed throughout the development and research phases of the project. The software development methodology and the technologies implemented played an integral part of the project.

\section{ Software Development Methodology} 
The software development methodology used for this project is an agile approach, mainly the Scrum framework. The Scrum implementation is used to the deliver the social media platform and to understand what can be the most effective in terms of delivering the social media project \cite{garzaniti2019effectiveness}. Agile methodologies focus on iterative and incremental development, allowing flexibility and responsiveness to any change request. The project had to meet many requirements to further develop the social media application. Many features have been taken into consideration in order to enhance the project and make it more attractive for users. The agile approach made the project more organized and reduced the risk of the loss of control. Scrum, with its iterative sprints and regular feedback loops, demonstrates the importance of managing the dynamic aspects of developing a social media platform. The social media platform had features to achieve during the coding. New requirements had to be adapted to ensure smooth during the development. The agile approach has proven to be a reliable step towards achieving certain functions. Project development involved the use of sprints to control and organize the development. During the sprints, the tasks selected for the project were processed. All issues related to the progression of the development have been studied. The development was reviewed to ensure the project was on track and functioning properly. The goal of the sprint was to deliver a social media application that allowed for users to communicate. Regular sprint reviews allowed the project to continuously improving and adapt to changing requirement and challenges each week.

\subsection{Agile Methodology}
Agile Methodology is an effective software development method. The iterative and incremental approach to software development focused on the flexibility and users need. It enabled dynamic planning and quick delivery of functional increments. Incremental requirements refine the design, coding, and testing at all phases of production activity \cite{kumar2012impact}. The goal of Agile Methodology is to develop high quality software in a short time, be self organizing, focus on user needs, require fewer documents and reduce the time to market \cite{kumar2012impact}. The requirements may change as the tasks are divided into multiple iterations to create a release plan that advance the project through the iteration phase \cite{altameem2015impact}. The product backlog contains all the bug fixes, requirements, features, and non-functional requirements that an effective software must address \cite{altameem2015impact}.

Adaptability is very beneficial to project management because agile embraces changes in requirements so that to allow a developer can respond to changing project requirements. Agile methodology has emerged to serve as a replacement for traditional methods of software development methods that were very limited and predictable \cite{altameem2015impact}. Agile methodology ensures a developer can handle various aspect of development, including coding, testing and deployment. Agile ensures that the product meets user expectations. The project focuses on the needs of user of a social media platform. Agile methodology promotes trust, improves the trust between the developers and users and shows that it is beneficial for a project \cite{altameem2015impact}.

\subsection{Scrum}
Scrum is a very popular framework within the agile methodology. It provides a structured and flexible way to develop software. It contains many important elements that help developers improve their ability to create quality software application. Scrum contains a set of rules and responsibilities that never change in order to help the developer ensure that the quality, stability, time, cost and control of the product are in good condition \cite{darwish2016requirements}. Scrum ensures that a project is delivered in the best possible way because it helps developers solve complex problems \cite{darwish2016requirements}. User expectations from the project take priority when a project is delivered for publication. A product backlog, sprints, and increments manage project tasks and bugs to create a project release. With the help of the Scrum framework, project work can be planned well. It provides a sprint retrospective to reflect on the sprint and look for opportunities for improvement. Scrum promotes transparency, adaptability and continuous improvement. 

\section{ Technologies Used} 
The technologies used for the social media platform help shape the blueprint to create a successful application. The technologies used for the project are diverse and include a range of tools that adapt to the requirement and objectives of the project.

\subsection{Frontend Technologies}
\subsubsection{React.Js}
The frontend of the social media platform is developed using the React library. React is really focused on project delivery thanks to its modular and reusable UI components, made possible by a component-based architecture. React has provided a responsive and interactive user interface. React ensured a robust and versatile social media platform. The use of React is an important aspect of the project's attractive application features. React is a reliable programming technology if you need to develop a dynamic user experience for a social media application. The functions and features of the project were all developed using the components used to develop them. React made it possible to seamlessly implement design the social media application design with these features.

\subsubsection{Material-UI}
The style and visuals representation of the project was handled using the reliable technology called Material-UI. Material-UI ensured a consistent and aesthetically pleasing design. The React UI framework provides a set of predefined components that follow the principles of Material-UI. The use of Material-UI was very consistent across use the rest of the social media platform.

\subsubsection{Dropzone}
Uploading images from the project was handled very well with Dropzone. React Dropzone is integrated for image upload usage. Dropzone is a way for users to upload directly from their local file system to a  designated area on the web page. Dropzone is useful for having images on the social media platform. 

\subsubsection{Formik and Yup}
Formik and Yup are very popular libraries in React and they are often used together for form management and validation. The purpose of Formik is to help manage form status and manage form submissions. Formik offered a dynamic form for user login and registration. The purpose of Yup is for value parsing and validation. Yup works with Formik for defining validation rules for form fields.

\subsubsection{React Router}
React Router is a useful library for handling the navigation in the social media application. This allowed users to switch between different parts of the application without having to reload the entire page.

\subsubsection{Redux}
The project included a way to manage state called Redux. State management in the frontend managed by Redux, a predictable state container for the social media application. Redux has been implemented for efficient state management in the frontend. Redux ensured the stability and maintainability of the frontend. It has provided access to the efficiency of state management to manage complex interactions within the social media application.

Redux state management is used to ensure that the web application is scalable and maintainable, especially when dealing with difficult state interactions. Redux aims to predictably handle state mutation without losing the performance benefits of asynchronous and making the application is easier to test \cite{bugl2017learning}. Redux was very efficient in controlling and organizing the different states in the social media platform. It is effective in ensuring that the application meets the projects objectives. The main concepts of Redux include Store, Action, Reducer and Dispatch. The Store is a JavaScript object that stores state of the application because it is a container containing the state tree. The action describes the application state change. Reducers specify changes in application state changes in response to actions. Dispatch is a method that is used to dispatch actions that trigger the reducer to execute for the new state.

\subsection{Backend Technologies}
\subsubsection{Node.Js}
The backend server of the social media platform is built using the Node.js, known for its scalability. Node.js made it possible to handle a large number of simultaneous connections, which is crucial for the social media platform. Node.js enabled robust server-side development. Node.js is ideal for building scalable and high performance web applications, and  is often used with Express.js to develop web servers and applications. 

\subsubsection{Express.Js}
Express.js is a flexible Node.js web application framework that supports developers in building robust web servers. Express.js makes it easier for developers to create web applications and APIs by providing an organized structure. Express can be used to create APIs with the endpoints that handle various HTTP methods, making it a very popular tool for creating backends. Express has an efficient routing system that allows developers to define routes based on URL and HTTP methods. This made it easier to create users authentication and posts on the social media application. Express.js has also taken an approach with its middleware system that allows developers to extend the functionality of the web applications. You can add Middleware function to perform various tasks such as error handling and authentication.

\subsubsection{Bcrypt}
Bcrypt is a password hashing function that is used to securely store passwords in a database. It is designed to be adaptable, ensuring a strong password hashing option. Bcrypt is the best choice if you are implementing a password storage instead of trying to create a custom one. On the social media platform, bcrypt is used for secure hashing of passwords during the user registration and authentication during the user login. Bcrypt is useful for protecting user passwords in the database by storing them in a hash format with additional features such as salting.

\subsubsection{JSON Web Token}
JSON Web Token (JWT) is used in the project for security purposes. The
social media platform must to be secure, and JWT is used to add security to the user's information. JWT is used for user authentication, posts and secure user data. Its very important to handle the JWT securely and store them properly. A middleware function is implemented in the project which checks if a JWT is present in the header of an incoming request. If a valid token is found, it is validated using the JSON web token library. If the verification succeeds, the decoded payload is attached to the user request object. The middleware is designed to be inserted into the middleware chain before routes that require authentication. The middleware ensures that only requests with valid JWT can proceed to the protected routes, and is added to the decoded user information for further processing.

\subsubsection{mongoose}
Mongoose is an object data modeling library for Node.js and MongoDB. It provides a simple schema-based solution for modeling web application data and connecting to MongoDB. Mongoose makes it easy to communicate with MongoDB by providing a structured way to design schemas and perform CRUD (create, read, update and delete) functions. 

\subsection{Database}
\subsubsection{MongoDB}
The database used for the social media platform was the MongoDB. MongoDB is a well-known open source NoSQL database management system. MongoDB was chosen for this project because it offers flexibility, scalability and good performance. MongoDB is a suitable choice for building a social media platform. MongoDB is widely used in web applications and other scenarios where a dynamic and scalable database solution is required. MongoDB was used to store user data, posts, and other relevant information in a format that fits well with the flexibility of a social media platform. Data could be retrieved from the MongoDB to display on the platform.

\subsection{Project Management}
\subsubsection{Jira}
Jira is a very popular project management tool that is used to help developers to plan, track, and manage their work. Jira is widely used in software development and helps manage the social media platform creation process. Jira allows a developer to create and track issues, tasks, and other work items. It supports project planning and task assignment for the social media platform.