\chapter{Technology Review}
This chapter discusses the technological landscape of social media platforms and the specific technologies used in the ConnectSphere project. It was necessary to research the design need of the social media platform for the user experience. This review provides a foundation for the understanding the choices made in the development process and how they connect the goals of social media applications. Frontend and backend were important aspects to develop features for the applications. Jira was the project management tool to help manage the project and MongoDB for data storage. Other tools used allowed the developer to build the social media platform in order to achieve the set goals.

\section{Social Media}
Since the creation of social networks in 1997, the main purpose of social media application has been to facilitate people in terms of social connectivity \cite{shabbir2016impact}. The social media platforms provides people with natural ways to help them enjoy their lives and stay connected, as well providing different types of information, but it is 
 also a place where users can create media and content \cite{shabbir2016impact}. These aspects for are taken into consideration in the development of the project to promote users involvement, and create a seamless experience environment. The goal of the project is to create features such as posts, likes, images, comments and user interactions in order to develop a compelling user experience that encourages active participation. The motivation for any social media platform like Facebook, Twitter, and Instagram is to connect with people who share to similar interests \cite{shabbir2016impact}. Social media applications are all about constant communication to create a community for users, and maintaining that communication is a crucial part \cite{jain2014application}. 

 \section{React.js}
 React is a powerful frontend technology that is used to develop web applications \cite{rawat2020reactjs}. The project used the React framework to develop the frontend of the social media application using JavaScript. React is a reliable technology with the numerous advantages that make it easier to develop an attractive web application. React offers incredible performance, especially with the JSX, where JavaScript and XML works very well together \cite{rawat2020reactjs}. React is a popular framework that offers many other features to design the best user interface for a social media platform. React offers dynamic features like routing where the web applications are rendered on the machine \cite{rawat2020reactjs}. React provides the Redux library and this feature brings many enhancements within the components to render them when needed \cite{rawat2020reactjs}. React Redux also produces a container function as a wrapper component to determine with the way towards collaboration path with the store \cite{rawat2020reactjs}.

 React can be integrated with the Node.js backend using many APIs such as the RESTful API \cite{ishaq2023design}. React allows developers to efficiently create reusable UI components and manage the state of web applications \cite{ishaq2023design}. The API used in the project could 
 able to work smoothly with the backend to retrieve and store information about users and their content from the database. React provides hooks that allow developers to write class-based components by handling state management from functional components \cite{aryal2020mern}. The main purpose of hooks is to solve problems when writing functional components that have access to state, context and life-cycle methods \cite{aryal2020mern}.

\section{Node.js and Express.js}
In Node.js, developers create business logic in the backend of a web application \cite{shah2017node}. Node.js is a popular open source web server that allows developers to create dynamic web applications \cite{aryal2020mern}. Node.js can handle many requests by running a single thread \cite{shah2017node}. Node.js requires developers to know only a single language, such as JavaScript, which is an advantage it has over other backend frameworks \cite{shah2017node}. Node.js is highly  performant, and it uses JavaScript because JavaScript provides first-class functions and closures \cite{syed2014beginning}. Node.js allows developers to create models that are abstractions of the data in the MongoDB \cite{ishaq2023design}. 

Node.js is known for creating a single page application and adding features to it \cite{shah2017node}. Node.js is a bit difficult for developers to learn, but the learning curve is not that steep \cite{shah2017node}. Setting up event-driven functionality of feature is also a challenge for some developers, but Node.js is has the advantage of easily configuring your hardware or server in an organized way \cite{shah2017node}. Before Node.js it was difficult to learn languages to build on the server and client side, now developers can build powerful and fast applications for a MERN stack with Node.js \cite{shah2017node}.
Express.js is a minimal and dynamic Node.js web application framework that provides developers with functionality to build various web applications in a MERN stack \cite{ishaq2023design}. Express.js is clear and easy for developers to use because it provides an API for routes and handling HTTP request and middleware \cite{ishaq2023design}. Express.js provides middleware features with requests, response and a middleware function. The middleware breaks the request handler into multiple steps and run code to make changes to the request and response objects \cite{peters2017building}. Express.js has built-in functionality to handle incoming HTTP requests with a route associated with the request handler with the primary purpose of performing CRUD operations \cite{peters2017building}. Express has a helper function that sends static files, such as HTML and CSS files, to client side
\cite{peters2017building}. Both Express.js and Node.js are powerful server-side technologies that offer many advantages. There were few problems to with its operation, but there weren't many problems.

\section{MongoDB}
MongoDB is a NoSQL database that offers developers a scalable and flexible way to manage data \cite{ishaq2023design}. MongoDB provides persistence for app data and designed is scalability and developers \cite{aggarwal2018comparative}. This is a positive for developers building an application that needs a maintainable database. MongoDB solves the gap issues between key-value pair stores, which are quick and scalable databases \cite{aggarwal2018comparative}. MongoDB allows you to use of internal memory to store working sets for faster access to data \cite{aggarwal2018comparative}. Other benefits offered by MongoDB include clear object structure, simple join, tuning, schema-free, and deep quires \cite{aggarwal2018comparative}. These are some of the reasons why MongoDB is popular when creating a project to store data in database. There are some drawbacks to using MongoDB. The disadvantages of MongoDB are the gigantic data size, less flexibility in query execution and lack of
transactional support \cite{aggarwal2018comparative}. MongoDB is a document-oriented database with no joins or transactions to make writing queries easier \cite{zhao2013modeling}.

MongoDB supports indexing, allowing you to index to any attributes in the database, and secondary indexes are also available \cite{zhao2013modeling}. Replication in MongoDB has support for master and slave replication, which provides redundancy, backup and automatic fail over \cite{zhao2013modeling}. Load balancing in MongoDB scales horizontally by using partitioning to distribute only a single logical database system across a group of machines \cite{zhao2013modeling}. MongoDB has the ability to be used as a file system that can store any file using the GridFS functionality \cite{zhao2013modeling}. MapReduce is supported by MongoDB and allows users to retrieve results using the SQL group-by clause \cite{zhao2013modeling}. All these additional features supported by MongoDB show how reliable it is for developers to use for when managing their database while building an application.

\section{JSON Web Token}
It is very important for user authentication and access management in web applications because it is necessary for network security \cite{ahmed2019authentication}. JSON Web Token (JWT) is a JSON object that provides a secure approach to representing a set of information between two parties \cite{ahmed2019authentication}. JWT represents the claims format for restricting the scope of an environment such as HTTP authorization headers and URI request parameters \cite{haekal2016token}. The claim is encode by JWT to be sent as a JSON object that is used as a payload, so the claims can to be a digitally signed message authentication code \cite{haekal2016token}. The server side uses JWT for queries until users log out of the application \cite{ahmed2019authentication}. One of the main functions of JWT is to enable state transition for communication, while a client cannot modify the information contained in the token \cite{ahmed2019authentication}. The JWT header has information about the JWT signature to be computed, the JWT payload has the data stored in the JWT, which could be an id or email, and the JWT signature has the signature of the header and payload \cite{ahmed2019authentication}.

Clients sends a username and password authentication request that is verified by an authentication service that creates a JWT based on the retrieved user authorization data \cite{ahmed2019authentication}. The token is used by the client with the request to a secure resource on the server, and the server receives the request to extract the user's authentication data from the JWT to send a response based on a valid or invalid token \cite{ahmed2019authentication}. implementing of token-based authentication using JWT on RESTful web services as an alternative to replacing server-based authentication \cite{haekal2016token}. JWT is considered a scalable solution providing performance benefits for user access control in large scale systems and there is a need to provide a general solution for all types of applications \cite{ahmed2019authentication}. JSON Web Token provides the security needed to build a web application.

\section{Socket.IO}
Socket.IO is similar to a web socket in that it provides the ability for real-time communication between the client and the server \cite{gelens2014gevent}. The idea of Socket is that it is a virtual socket, which abstracts from the fact that the transport does long-polling, and provides other functions similar to web sockets for all \cite{gelens2014gevent}. When building a unique web application, Socket.IO provides developers with reliable technology to create features like notifications and a live chat. Socket provides two way communication that connecting two programs running over the network \cite{joby2016socket}. The basic steps to create the socket are to open the socket, wait for the client, establish communication with the I/ O flow, perform communication and then finally close the socket \cite{joby2016socket}. This allows for real-time communication between the server and the client to really be efficient. With Socket.IO, a chat function can be managed efficiently,
enabling a flexible and modular communication model. Socket.Io makes it easy to implement a notification functionality with real-time communication between client and server to produce instant updates. Socket.IO is a powerful tool for creating a real-time communication web application with fast and efficient communication between server and client. 

\section{Jira}
To create an efficient applications, project development necessitates careful planning based on objectives and goals. Developer's work heavily relies on project management. Jira is a popular and useful tool for project management. Atlassian developed Jira as an issues tracking system in 2002 \cite{fisher2013utilizing}. Jira provides developers with a variety of tools for local customization to meet project requirements,
including issue types, workflows, and user stories \cite{fisher2013utilizing}. When an issues is submitted, a member of the project team automatically notified and assigned to a specific task of the project \cite{fisher2013utilizing}. Assigning tasks to developers for specific tasks in the project allows for a clear focus on achieving the goals of application development. Review teams conduct design and code reviews to assess the complexity of software effort, aiming to ensure software quality assurance \cite{fisher2013utilizing}. Jira ensures that developers focus on the coding of the project in order to stay organized with the application requirements. Jira lacks the capability to regulate the modifications made to the specific fields in a Jira issue, allowing only authorized users to make changes if the specific fields are available on the entry screen \cite{fisher2013utilizing}.

Jira has a custom Scrum board and dynamic Kanban board, which seamlessly integrate with popular environments, are essential components of its functionality \cite{ozkan2019agile}. In addition, Jira provides bug tracking, flow diagrams, custom developer tools, and issue and reporting capabilities \cite{ozkan2019agile}. Jira has established itself as a trusted tool that is widely used by developers and large organizations \cite{ozkan2019agile}.  

\section{Render}
The ConnectSphere social media application is scheduled for deployment once the completing the design and functionality testing are finalized. Render is the tool utilized to deploy the web application. Render is a cloud application that enables the deployment of MERN stack applications, emphasizing the important of having a reliable deployment mechanism. Cloud applications offer a robust computing platform to build a application on top of the platform \cite{fylaktopoulos2016overview}. The application developer is solely responsible for developing the application, while the cloud application take care of maintaining the platform \cite{fylaktopoulos2016overview}. Render provides a variety of services, including databases, Secure Socket Layer certificates and domains. Render allows developers to easily deploy and scale their web applications across a large infrastructure, without having to worry about the server management and scaling \cite{ssemakula2023low}. GitHub, GitLab and Docker are popular development tools that integrate with Render to make it easy to deploy an application and manage the infrastructure \cite{ssemakula2023low}.